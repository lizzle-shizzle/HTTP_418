%%%%%%%%%%%%%%%%%%%%%%%%%%%%%%%%%%%%%%%%%
%  My documentation report
%  Objetive: Explain what I did and how, so someone can continue with the investigation
%
% Important note:
% Chapter heading images should have a 2:1 width:height ratio,
% e.g. 920px width and 460px height.
%
%%%%%%%%%%%%%%%%%%%%%%%%%%%%%%%%%%%%%%%%%

%----------------------------------------------------------------------------------------
%	PACKAGES AND OTHER DOCUMENT CONFIGURATIONS
%----------------------------------------------------------------------------------------

\documentclass[11pt,fleqn]{book} % Default font size and left-justified equations

\usepackage[top=3cm,bottom=3cm,left=3.2cm,right=3.2cm,headsep=10pt,letterpaper]{geometry} % Page margins

\usepackage{xcolor} % Required for specifying colors by name
\definecolor{darkgreen}{RGB}{0, 153, 0} % Define the orange color used for highlighting throughout the book

% Font Settings
\usepackage{avant} % Use the Avantgarde font for headings
%\usepackage{times} % Use the Times font for headings
\usepackage{mathptmx} % Use the Adobe Times Roman as the default text font together with math symbols from the Sym­bol, Chancery and Com­puter Modern fonts

\usepackage{microtype} % Slightly tweak font spacing for aesthetics
\usepackage[utf8]{inputenc} % Required for including letters with accents
\usepackage[T1]{fontenc} % Use 8-bit encoding that has 256 glyphs

% Bibliography
\usepackage[style=alphabetic,sorting=nyt,sortcites=true,autopunct=true,babel=hyphen,hyperref=true,abbreviate=false,backref=true,backend=biber]{biblatex}
\addbibresource{bibliography.bib} % BibTeX bibliography file
\defbibheading{bibempty}{}

\input{structure} % Insert the commands.tex file which contains the majority of the structure behind the template

\begin{document}

%----------------------------------------------------------------------------------------
%	TITLE PAGE
%----------------------------------------------------------------------------------------

\begingroup
\thispagestyle{empty}
\AddToShipoutPicture*{\put(0,0){\includegraphics[scale=1]{frontcover}}} % Image background
\centering
\vspace*{5cm}
\par\normalfont\fontsize{35}{35}\sffamily\selectfont
\textbf{Harvest}\\
{\LARGE User Manual for Harvest System}\par % Book title
\vspace*{0.5cm}
{\Huge HTTP\_418}\par
\centering
\vspace*{0.5cm}
\begin{itemize}[label={}, noitemsep]	
		\Large
		\item \begin{center} Christiaan Saaiman, 12059138 \end{center}
		\item \begin{center} Michael Loosen, 14017254 \end{center}
		\item \begin{center} Elizabeth Bode, 14310156 \end{center}
		\item \begin{center} LC Meyers, 14024633 \end{center}	
\end{itemize}
\endgroup

%----------------------------------------------------------------------------------------
%	TABLE OF CONTENTS
%----------------------------------------------------------------------------------------

\chapterimage{orchard.png} % Table of contents heading image

\pagestyle{empty} % No headers

\tableofcontents % Print the table of contents itself

%\cleardoublepage % Forces the first chapter to start on an odd page so it's on the right

\pagestyle{fancy} % Print headers again

%----------------------------------------------------------------------------------------
%	CHAPTER 1
%----------------------------------------------------------------------------------------

\chapterimage{mangoes.png} % Chapter heading image

\chapter{Introduction}
	\section{About Harvest}
		*Short explanation/description about Harvest*
	\section{Installing Harvest}
		*Easy step-by-step guide to installing Harvest*
	\section{Creating an Account on Harvest}
		*Simple step-by-step guide to creating a user account on Harvest*

%----------------------------------------------------------------------------------------
%	CHAPTER 2
%----------------------------------------------------------------------------------------
\chapterimage{macadamias.png}

\chapter{Using Harvest on your Farm}
	\section{What do you Need to Know as a Farmer?}
		% So you have an account John Deere, what now? *Insert app usage with foreman in the field explanation*
		
		Harvest enables you to monitor your farm in its entirety from your home. You simply need to register your farm on the website, log into the website using your credentials and then everything you need to know regarding your farm can be accessed. This includes monitoring your foremen's behaviour, monitoring the performance of your workers overall and at certain times during the day, mapping out the orchard blocks on your farm and monitoring the crop yields they bring in seasonally, annually or currently. The crop yield density can also be easily determined by viewing the heat map that can be generated for your farm. All the details of specific orchard blocks, such as the type of crop, the type of irrigation used, the frequency that cultivation takes place and the type of measurement used for determining yields, no longer needs to manually be recorded. Now everything can be viewed and modified in one place via a simple and user-friendly interface designed to suit your needs. The assignment of workers to specific foreman and allocating foremen to specific orchard blocks can also effortlessly be done using Harvest. All of this is simply done by assigning credentials to your foremen, which allows them to record and view their worker's performance. You can view all the worker performance data that your foremen have entered at any time assuming both you and your foreman are connected to the Internet. The use of unique credentials per foreman allows you to be able to monitor their whereabouts according to their GPS location. These credentials will be generated once you have added a new foreman to the system and then you are required to notify the foreman of these credentials.\newline\newline
		
		How does this benefit you in the long term? Harvest also allows for quick and easy generation of reports regarding worker performance, the amount of crop yield per orchard block, the revenue generated according to the yield collected in a specific season and the amount of time it takes to cultivate certain crops. This offers a multitude of benefits in regard to strategical planning for the future. Such planning includes, financial planning according to seasons, staff requirements and staff payment planning according to how much time is required for cultivating specific crops and planning where to plant certain crops according to higher yields collected in specific orchard blocks. The monitoring of worker performance can also highlight whether certain foremen aren't being strict enough with managing their workers or highlight the lack of crop yield collected either due to extreme weather conditions or them not working hard enough.\newline\newline
		
		As a farmer, you already have a lot of work on your hands monitoring large amounts of land and staff. Allow Harvest to make this process increasingly easier with the vast amount of benefits it offers.

		
	\section{Using Harverst on your Computer}
		So how do you actually know what is going on in the field? *Insert explanation about web interface*
	\section{Using Harvest to Manage your Staff}
		Harvest isn't just for your crops, yields and so, but for your people as well!! You can assign foreman to the mobile app that goes with the Harvest website. There the foreman can use the mobile app to monitor the worker performance of the workers that are assigned under them. The mobile app not only let's you monitor worker performance but also serves as a way to make sure your foreman do not leave the farm when they should not. The app features GPS tracking that can notify you through the website when a foreman leaves his post.
		
		*insert step by step guide and screenshots*


%----------------------------------------------------------------------------------------
%	CHAPTER 3
%----------------------------------------------------------------------------------------

\chapterimage{avocados.png}
\chapter{Troubleshooting}
	\section{When my App won't connect?}
		*Come on guys, do I really need to say what comes here? :P*
	\section{When my Profile won't display?}
		*This is getting ridiculous, seriously*
	\section{When my Reports don't generate?}
		*Oh my word, the titles are screaming at you!!*
%----------------------------------------------------------------------------------------
%	CHAPTER 4
%----------------------------------------------------------------------------------------

\chapterimage{litchis.png} % Chapter heading image

\chapter{Frequently Asked Questions[FAQ]}
	\section{App Related}
	\section{Computer Related}
	\section{Profile Related}
	
%----------------------------------------------------------------------------------------
%	CHAPTER 5
%----------------------------------------------------------------------------------------

\chapterimage{farmer.png} % Chapter heading image

\chapter{Still having Problems?}
	\section{Telephone}
		*Contact details*
	\section{Email}
		*Contact details*
	\section{Website}
		*Contact details*

%----------------------------------------------------------------------------------------
%	CHAPTER 6
%----------------------------------------------------------------------------------------

\chapterimage{mangoes2.png} %Chapter heading image

\chapter{Glossary of Terms}
	\section{I am a glossary, short and stout...}
		*Insert difficult terms, phrases and lingo here*

\end{document}
