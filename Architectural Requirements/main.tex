%%%%%%%%%%%%%%%%%%%%%%%%%%%%%%%%%%%%%%%%%
%  My documentation report
%  Objetive: Explain what I did and how, so someone can continue with the investigation
%
% Important note:
% Chapter heading images should have a 2:1 width:height ratio,
% e.g. 920px width and 460px height.
%
%%%%%%%%%%%%%%%%%%%%%%%%%%%%%%%%%%%%%%%%%

%----------------------------------------------------------------------------------------
%	PACKAGES AND OTHER DOCUMENT CONFIGURATIONS
%----------------------------------------------------------------------------------------

\documentclass[11pt,fleqn]{book} % Default font size and left-justified equations

\usepackage[top=3cm,bottom=3cm,left=3.2cm,right=3.2cm,headsep=10pt,letterpaper]{geometry} % Page margins

\usepackage{xcolor} % Required for specifying colors by name
\definecolor{darkgreen}{RGB}{0, 153, 0} % Define the orange color used for highlighting throughout the book

% Font Settings
\usepackage{avant} % Use the Avantgarde font for headings
%\usepackage{times} % Use the Times font for headings
\usepackage{mathptmx} % Use the Adobe Times Roman as the default text font together with math symbols from the Sym­bol, Chancery and Com­puter Modern fonts

\usepackage{microtype} % Slightly tweak font spacing for aesthetics
\usepackage[utf8]{inputenc} % Required for including letters with accents
\usepackage[T1]{fontenc} % Use 8-bit encoding that has 256 glyphs

% Bibliography
\usepackage[style=alphabetic,sorting=nyt,sortcites=true,autopunct=true,babel=hyphen,hyperref=true,abbreviate=false,backref=true,backend=biber]{biblatex}
\addbibresource{bibliography.bib} % BibTeX bibliography file
\defbibheading{bibempty}{}

\input{structure} % Insert the commands.tex file which contains the majority of the structure behind the template

\begin{document}

%----------------------------------------------------------------------------------------
%	TITLE PAGE
%----------------------------------------------------------------------------------------

\begingroup
\thispagestyle{empty}
\AddToShipoutPicture*{\put(0,0){\includegraphics[scale=1]{frontcover}}} % Image background
\centering
\vspace*{5cm}
\par\normalfont\fontsize{35}{35}\sffamily\selectfont
\textbf{Harvest}\\
{\LARGE Software Architectural Requirements and Design}\par % Book title
\vspace*{0.5cm}
{\Huge HTTP\_418}\par
\centering
\vspace*{0.5cm}
\begin{itemize}[label={}, noitemsep]	
		\Large
		\item \begin{center} Christiaan Saaiman, 12059138 \end{center}
		\item \begin{center} Michael Loosen, 14017254 \end{center}
		\item \begin{center} Elizabeth Bode, 14310156 \end{center}
		\item \begin{center} LC Meyers, 14024633 \end{center}	
\end{itemize}
\endgroup

%----------------------------------------------------------------------------------------
%	TABLE OF CONTENTS
%----------------------------------------------------------------------------------------

\chapterimage{orchard.png} % Table of contents heading image

\pagestyle{empty} % No headers

\tableofcontents % Print the table of contents itself

%\cleardoublepage % Forces the first chapter to start on an odd page so it's on the right

\pagestyle{fancy} % Print headers again

%----------------------------------------------------------------------------------------
%	CHAPTER 1
%----------------------------------------------------------------------------------------

\chapterimage{mangoes.png} % Chapter heading image

\chapter{Vision}


%----------------------------------------------------------------------------------------
%	CHAPTER 2
%----------------------------------------------------------------------------------------
\chapterimage{macadamias.png}

\chapter{Background}
\begin{itemize}
	\item Limitations of current system\newline\newline
	The current system being used by the client involves manually tracking worker performance with the use of a clipboard, with a list of all the workers names, and a pen.  The farmers have to manually record every yield that each worker collects in order to track their performance to determine their wages. The current system is inefficient due to the tedious nature of it. This tediousness reduces the foreman’s productivity as time is wasted unnecessarily searching for workers in the list and keeping track of all the lists chronologically. As the foreman is in charge of the workers, the workers have to wait until the yield they collected has been recorded which in turn also reduces their productivity.\newline
	
	\item High theft issue\newline\newline
	Farmers in the subtropical growers industry are continuously burdened by theft of their products by their own employees. This results in decreased revenue, increased costs, reduced productivity and a decrease in employer-employee trust, which creates the potential for a poor working environment. All of which necessitates the development of a new system that seeks to reduce the probability of crime by implementing monitoring on the foreman’s devices.\newline
	
	\item Need for statistical data\newline\newline
	The potential for generating statistical data that could aid farmers in future decisions from the manual data they collect is immense. The current paper-based storage of data prevents the generation of statistical reports due to the immense workload and time it requires. There is a huge need for reports indicating the highest producing orchard blocks, appropriate allocation of orchard block conditions to crop types and expected revenue. All of which could assist the farmers in future strategic planning.
\end{itemize}
%----------------------------------------------------------------------------------------
%	CHAPTER 3
%----------------------------------------------------------------------------------------

\chapterimage{avocados.png}
\chapter{Architecture Requirements}

	\section{Architectural Scope}
	\section{Access Channel Requirements}
		\subsection{Human Access Channels}
		\subsection{System Access Channels}

	\section{Quality Requirements}
		\subsection{Performance}
		\subsection{Reliability}
		\subsection{Scalability}
		\subsection{Security}
		\subsection{Flexibility}
		\subsection{Maintainability}
		\subsection{Auditability/Monitorability}
		\subsection{Integrability}
		\subsection{Cost}
		\subsection{Usability}
		
	\section{Integration Requirements}
	
	\section{Architecture Constraints}
		\subsection{Architectural Patterns}

%----------------------------------------------------------------------------------------
%	CHAPTER 4
%----------------------------------------------------------------------------------------

\chapterimage{litchis.png} % Chapter heading image

\chapter{Architectural Tactics or Strategies}

\section{Software Engineering Context : Architectural Tactics}
\section{Performance Tactics}
	\subsection{Three Groups of Performance Tactics}
	\subsection{System Context}
\section{Reliability Tactics}
	\subsection{System Context}
\subsection{Scalability Tactics}
	\subsection{System Context}
\subsection{Security Tactics}
	\subsection{System Context}
\subsection{Flexibility Tactics}
	\subsection{System Context}
\subsection{Maintainability Tactics}
	\subsection{System Context}
\subsection{Auditability Tactics}
	\subsection{System Context}
\subsection{Integrability and Extensibility Tactics}
	\subsection{System Context}
\subsection{Cost Tactics}
	\subsection{System Context}
\subsection{Usability Tactics}
	\subsection{System Context}
	
%----------------------------------------------------------------------------------------
%	CHAPTER 5
%----------------------------------------------------------------------------------------

\chapterimage{farmer.png} % Chapter heading image

\chapter{Reference Architectures and Frameworks}
	
	\section{Backend System}
		\subsection{Programming Languages}
		\subsection{Frameworks}
		\subsection{Libraries}
		\subsection{Database System}
		\subsection{Operating System}
		\subsection{Dependency Management and Build Tools}
	\section{Web Interface}
		\subsection{Programming Languages}
		\subsection{Frameworks}
		\subsection{Libraries}
		\subsection{Database System}
		\subsection{Operating System}
		\subsection{Dependency Management and Build Tools}
	\section{Android Client}
		\subsection{Programming Languages}
		\subsection{Frameworks}
		\subsection{Libraries}
		\subsection{Database System}
		\subsection{Operating System}
		\subsection{Dependency Management and Build Tools}	
	
	
%----------------------------------------------------------------------------------------
%	CHAPTER 6
%----------------------------------------------------------------------------------------

\chapterimage{mangoes2.png} % Chapter heading image

\chapter{Access and Integration Channels}

	\section{Access Channels}
	\section{Integration Channels}
	
	
%----------------------------------------------------------------------------------------
%	CHAPTER 7
%----------------------------------------------------------------------------------------

\chapterimage{macadamias2.png} % Chapter heading image

\chapter{Technologies}

	\section{Backend System}
		\subsection{Programming Languages}
		\subsection{Frameworks}
		\subsection{Libraries}
		\subsection{Database System}
		\subsection{Operating System}
		\subsection{Dependency Management and Build Tools}
	\section{Web Interface}
		\subsection{Programming Languages}
		\subsection{Frameworks}
		\subsection{Libraries}
		\subsection{Database System}
		\subsection{Operating System}
		\subsection{Dependency Management and Build Tools}
	\section{Android Client}
		\subsection{Programming Languages}
		\subsection{Frameworks}
		\subsection{Libraries}
		\subsection{Database System}
		\subsection{Operating System}
		\subsection{Dependency Management and Build Tools}

%----------------------------------------------------------------------------------------
%	CHAPTER 8
%----------------------------------------------------------------------------------------

\chapterimage{avocados2.png} % Chapter heading image

\chapter{Functional Requirements and App Design}

	\section{Use Case Prioritization}
	\section{Use Cases}
	
	
%----------------------------------------------------------------------------------------
%	CHAPTER 9
%----------------------------------------------------------------------------------------

\chapterimage{litchiTree.png} % Chapter heading image

\chapter{Open Issues}

\section{I am the first Open Issue}

\end{document}