%%%%%%%%%%%%%%%%%%%%%%%%%%%%%%%%%%%%%%%%%
%  My documentation report
%  Objetive: Explain what I did and how, so someone can continue with the investigation
%
% Important note:
% Chapter heading images should have a 2:1 width:height ratio,
% e.g. 920px width and 460px height.
%
%%%%%%%%%%%%%%%%%%%%%%%%%%%%%%%%%%%%%%%%%

%----------------------------------------------------------------------------------------
%	PACKAGES AND OTHER DOCUMENT CONFIGURATIONS
%----------------------------------------------------------------------------------------

\documentclass[11pt,fleqn]{book} % Default font size and left-justified equations

\usepackage[top=3cm,bottom=3cm,left=3.2cm,right=3.2cm,headsep=10pt,letterpaper]{geometry} % Page margins

\usepackage{xcolor} % Required for specifying colors by name
\definecolor{darkgreen}{RGB}{0, 153, 0} % Define the orange color used for highlighting throughout the book

% Font Settings
\usepackage{avant} % Use the Avantgarde font for headings
%\usepackage{times} % Use the Times font for headings
\usepackage{mathptmx} % Use the Adobe Times Roman as the default text font together with math symbols from the Sym­bol, Chancery and Com­puter Modern fonts

\usepackage{microtype} % Slightly tweak font spacing for aesthetics
\usepackage[utf8]{inputenc} % Required for including letters with accents
\usepackage[T1]{fontenc} % Use 8-bit encoding that has 256 glyphs

% Bibliography
\usepackage[style=alphabetic,sorting=nyt,sortcites=true,autopunct=true,babel=hyphen,hyperref=true,abbreviate=false,backref=true,backend=biber]{biblatex}
\addbibresource{bibliography.bib} % BibTeX bibliography file
\defbibheading{bibempty}{}

\input{structure} % Insert the commands.tex file which contains the majority of the structure behind the template

\begin{document}

%----------------------------------------------------------------------------------------
%	TITLE PAGE
%----------------------------------------------------------------------------------------

\begingroup
\thispagestyle{empty}
\AddToShipoutPicture*{\put(0,0){\includegraphics[scale=1]{frontcover}}} % Image background
\centering
\vspace*{5cm}
\par\normalfont\fontsize{35}{35}\sffamily\selectfont
\textbf{Harvest}\\
{\LARGE Testing Documentation}\par % Book title
\vspace*{0.5cm}
{\Huge HTTP\_418}\par
\centering
\vspace*{0.5cm}
\begin{itemize}[label={}, noitemsep]	
		\Large
		\item \begin{center} Christiaan Saaiman, 12059138 \end{center}
		\item \begin{center} Michael Loosen, 14017254 \end{center}
		\item \begin{center} Elizabeth Bode, 14310156 \end{center}
		\item \begin{center} LC Meyers, 14024633 \end{center}	
\end{itemize}
\endgroup

%----------------------------------------------------------------------------------------
%	TABLE OF CONTENTS
%----------------------------------------------------------------------------------------

\chapterimage{orchard.png} % Table of contents heading image

\pagestyle{empty} % No headers

\tableofcontents % Print the table of contents itself

%\cleardoublepage % Forces the first chapter to start on an odd page so it's on the right

\pagestyle{fancy} % Print headers again

%----------------------------------------------------------------------------------------
%	CHAPTER 1
%----------------------------------------------------------------------------------------

\chapterimage{mangoes.png} % Chapter heading image

\chapter{User Module}
	\section{Functionality which is Correctly Implemented}
		\begin{itemize}
			\item\textbf{Creating a farmer account} -
			The signup page for registering a farmer account contains input fields so that the user can input their necessary details. Each input field has certain criteria, which is tested for compliance to ensure incorrect data is not entered. The First Name and Last Name fields ensure that the user doesn't enter more than 50 characters. The Date of Birth field makes use of a date picker to ensure incorrect data cannot be entered. The Email Address field makes use of an email input, which ensures that only a valid email address can be entered. The Password and Confirm Password field only allow passwords of a minimum of 5 characters and maximum of 10 characters to be entered. The Confirm Password field ensures that the data entered into the Password field and the Confirm Password field match. Appropriate error messages are displayed if any incorrect or no data is entered. The Create Account button is disabled until all inputs have valid data entered into them. This ensures that none of the fields are empty and that all data entered is valid. Once everything is valid, the Create Account button is enabled. After submission of the form, if the email address entered is already taken, then an error message is displayed specifying that that email address already exists and then navigates back to the signup page. If submission is successful, the user will automatically be logged into the sysetm.
			
			\item\textbf{Viewing \& editing a farmer account} - 
			The edit form contains input fields that are populated with the user's current information. The First Name and Last Name fields ensure that the user doesn't enter more than 50 characters. The Date of Birth field makes use of a date picker to ensure incorrect data cannot be entered. The Email Address field makes use of an email input, which ensures that only a valid email address can be entered. There is a Change Password button, which navigates to the change password form. This functionality is specified under Change Password. The Update Details button is disabled until all inputs have valid data entered into them. This ensures that none of the fields are empty and that all data entered is valid. Once everything is valid, the Update Details button is enabled. After submission of the form, if the email address entered is already taken, then an error message is displayed specifying that that email address already exists and then navigates back to the edit profile page.
			
			\item\textbf{Logging in} -
			The Sign In form contains an inputs for the farmer's Email Address, Password and a Remember Me checkbox. There is also a link to recover your password. Both the email and password inputs are required and the Sign In button is disabled until these inputs have data entered into them. If the user has an account and enters in the correct details for this account then the user will be directed to their homepage. Otherwise, an error message will be displayed specifying that the email and password entered doesn't match any account on the system and they will remain on the landing page.
			
			\item\textbf{Logging out} -
			The user can only view the Log Out button if they have been logged into the system. When the user clicks the Log Out button, the user's session is destroyed and their information is reset to null causing them to be logged out of the system. Upon logout, they will be redirected to the landing page.
			
			\item\textbf{Changing password} -
			The user can only navigate to this form through the edit profile form, which is only accessible if the user is logged in. This ensures that the user does have an account present in the database, and eliminates the need to re-enter your current password. The Change Password form contains inputs for New Password and Confirm New Password. The Password and Confirm Password field only allow passwords of a minimum of 5 characters and maximum of 10 characters to be entered. 
			-%The Confirm Password field ensures that the data entered into the Password field and the Confirm Password field match. Appropriate error messages are displayed if any incorrect or no data is entered. 
			The Change Password button is disabled until all inputs have valid data entered into them. This ensures that none of the fields are empty and that all data entered is valid. Upon submission of the form, the user is returned to the edit profile page.
			
			\item\textbf{Recovering password} -
			The user can navigate to this form without being logged into the system. The user is requested to enter the email address they used to sign up their account. The form only submits successfully if the email address entered is in fact linked with an existing account. Once the user submits this form, an email is sent with a link with a token, which provides them access to the change password page. This token expires one hour after the email was sent. The user is then navigated to the homepage. After the link has been selected, the change password form displays and contains inputs for New Password and Confirm New Password. The Password and Confirm Password field only allow passwords of a minimum of 5 characters and maximum of 10 characters to be entered. The Change Password button is disabled until all inputs have valid data entered into them. This ensures that none of the fields are empty and that all data entered is valid. Upon submission of the form, the user is returned to the edit profile page.
		\end{itemize}
	\section{Short-comings of Implemented Functionality}
		\begin{itemize}
			\item\textbf{Creating a farmer account} -
			The Date of Birth field needs to ensure that a range of invalid dates is excluded, such as this year as the year of birth. This functionality still needs to be implemented. If submission of the form is unsuccessful, there is no error message displayed specifying that the signup failed. However, a redirect does return the user to the signup page. There is currently no functionality included to specify that this account has superuser rights.
			
			\item\textbf{Viewing \& editing a farmer account} - 
			If the updating of the account is unsuccessful an error message should be dispayed and the user should be redirected to the edit profile page. If the update is successful, a message specifying this success should be displayed and the user should be redirected to their homepage.
			
			\item\textbf{Logging in} -
			There is currently no functionality specifying whether user account has been locked or not, such as if a foreman no longer works for a farmer then their details would need to be archived. The remember me functionality hasn't been implemented yet as it is a nice-to-have component of the system. The recovery password link's functionality is specified under the Recover Password use case.
			
			\item\textbf{Logging out} -
			There are no short-comings with this functionality.
			
			\item\textbf{Changing password} -
			The error handling for invalid data and the Password and Confirm Password aren't currently working. If an error occurs with the updating of the password field, then an error message should be displayed specifying the error that occurred. Once the password has been changed, a message should display to indicate success.
			
			\item\textbf{Recovering password} -
			The success message to indicate that the email has successfully been sent currently doesn't display. The error handling for invalid data and the Password and Confirm Password aren't currently working. If an error occurs with the updating of the password field, then an error message should be displayed specifying the error that occurred. Once the password has been changed, a message should display to indicate success.
		\end{itemize}
	\section{Missing Functionality}
		\begin{itemize}
			\item\textbf{Importing census data} -
			This feature hasn't been implemented yet as it is considered  an important functionality and will implemented at a later stage.
		\end{itemize}
	\section{Unit Test Report for User Module}
		The unit tests for this module are currently incomplete. Once complete, this section will show a screenshot of the results of the unit tests, along with an evaluation of these results.

%----------------------------------------------------------------------------------------
%	CHAPTER 2
%----------------------------------------------------------------------------------------
\chapterimage{macadamias.png}

\chapter{Farm Module}
	\section{Functionality which is Correctly Implemented}
		\begin{itemize}
			\item\textbf{Adding a farm} -
			The create page for farm contains input fields that the user uses to input the required data. Each input field has certain criteria, which is tested for compliance to ensure incorrect data is not entered. Farm Name, Registered Farm Name and Company Name requires the user to enter no more than 120 characters. Farm Name is optional. Farm Size is optional and requires the user to enter numerical data. Province uses a drop down list to limit the user to only the 9 provinces we have in our country. Region, Farming Region and Portion Number require the user to enter no more than 50 characters. Region and portion Number are optional. Error messages are displayed is any incorrect data is entered or no data is entered where required. the Create Farm button is disabled of any of the inputs are invalid. This ensures that the user does not enter any invalid data. If submition is successfull the user is taken to the main page where the details of the farm are displayed.
			
			%\item\textbf{Viewing a farm} -
			%Some description.
			
			\item\textbf{Editing a farm} -
			Editing a farm contains the same inputs and validation as when creating a farm. This ensures that no invalid data is stored when updating a farm. Once everything is valid, the Update Details button is enabled. Apon successfull submition the user is taken back to the main page where the updated farm details are reflected.
			
			\item\textbf{Adding a orchard block} -
			Some description. 
			
			\item\textbf{Viewing a orchard block} -
			Some description.
			
			\item\textbf{Editing a orchard block} -
			Some description.
		\end{itemize}
	\section{Short-comings of Implemented Functionality}
		\begin{itemize}
			\item\textbf{Adding a farm} -
			There is no way to pass any gracefull error messages to the user if submition fails.
			
			\item\textbf{Viewing a farm} -
			The user cannot select between different farms.
			
			\item\textbf{Editing a farm} -
			There is no way to pass any gracefull error messages to the user if submition fails.
			
			\item\textbf{Adding a orchard block} -
			Some description. 
			
			\item\textbf{Viewing a orchard block} -
			Some description.
			
			\item\textbf{Editing a orchard block} -
			Some description.
		\end{itemize}
	\section{Missing Functionality}
		\begin{itemize}
			\item\textbf{Viewing heat map} -
			Some description.
		\end{itemize}
	\section{Unit Test Report for Farm Module}
		The unit tests for this module are currently incomplete. Once complete, this section will show a screenshot of the results of the unit tests, along with an evaluation of these results.


%----------------------------------------------------------------------------------------
%	CHAPTER 3
%----------------------------------------------------------------------------------------

\chapterimage{avocados.png}
\chapter{Employee Module}
	\section{Functionality which is Correctly Implemented}
		\begin{itemize}
			\item\textbf{Adding a foreman} -
			Some description. 
			
			\item\textbf{Viewing a foreman} -
			Some description.
			
			\item\textbf{Editing a foreman} -
			Some description.
		\end{itemize}
	\section{Short-comings of Implemented Functionality}
		\begin{itemize}
			\item\textbf{Adding a foreman} -
			Some description. 
			
			\item\textbf{Viewing a foreman} -
			Some description.
			
			\item\textbf{Editing a foreman} -
			Some description.
		\end{itemize}
	\section{Missing Functionality}
		\begin{itemize}
			\item\textbf{Maintaining worker-foreman assignments} -
			Some description.
			
			\item\textbf{Viewing worker-foreman assignments} -
			Some description.
			
			\item\textbf{Viewing foreman-orchard Block allocations} -
			Some description.
		\end{itemize}
	\section{Unit Test Report for Employee Module}
		The unit tests for this module are currently incomplete. Once complete, this section will show a screenshot of the results of the unit tests, along with an evaluation of these results.

%----------------------------------------------------------------------------------------
%	CHAPTER 4
%----------------------------------------------------------------------------------------

\chapterimage{litchis.png} % Chapter heading image

\chapter{Reporting Module}
	\section{Functionality which is Correctly Implemented}
		\begin{itemize}
			\item\textbf{Nothing yet} -
			The entire reporting module is considered a nice-to-hve functionality as the bulk of the system's functionality and value doesn't rely heavily on reporting. Thus, we have decided to focus on the other more important functionailities and leave this one for after the bulk of the functionality is complete. Therefore, testing cannot be performed on this module at this time.
		\end{itemize}
	\section{Short-comings of Implemented Functionality}
		\begin{itemize}
			\item{Not applicable}
		\end{itemize}
	\section{Missing Functionality}
		\begin{itemize}
			\item\textbf{Generating a statistical report of worker performance} -
			Some description.
			
			\item\textbf{Generating a statistical report regarding crop yield per orchard} -
			Some description.
			
			\item\textbf{Generating a revenue report regarding seasonal yields} -
			Some description.
			
			\item\textbf{Generating a statistical report regarding time taken to yield specific crops} -
			Some description.
		\end{itemize}
	\section{Unit Test Report for Reporting Module}
		The unit tests for this module are currently incomplete. Once complete, this section will show a screenshot of the results of the unit tests, along with an evaluation of these results.
	
%----------------------------------------------------------------------------------------
%	CHAPTER 5
%----------------------------------------------------------------------------------------

\chapterimage{farmer.png} % Chapter heading image

\chapter{Notifications Module}
	\section{Functionality which is Correctly Implemented}
		\begin{itemize}
			\item\textbf{Nothing yet} -
			The entire notifications module is considered a nice-to-hve functionality as the bulk of the system's functionality and value doesn't rely heavily on notifications. Thus, we have decided to focus on the other more important functionailities and leave this one for after the bulk of the functionality is complete. Therefore, testing cannot be performed on this module at this time.
		\end{itemize}
	\section{Short-comings of Implemented Functionality}
		\begin{itemize}
			\item{Not applicable}
		\end{itemize}
	\section{Missing Functionality}
		\begin{itemize}
			\item\textbf{Viewing notifications} -
			Some description.
			
			\item\textbf{Deleting notifications} -
			Some description.
			
			\item\textbf{Notifying farmer regarding foreman’s locations} -
			Some description.
			
			\item\textbf{Notifying farmer of foreman’s activity history every half an hour} -
			Some description.			
		\end{itemize}
	\section{Unit Test Report for Notifications Module}
		The unit tests for this module are currently incomplete. Once complete, this section will show a screenshot of the results of the unit tests, along with an evaluation of these results.
	
	
%----------------------------------------------------------------------------------------
%	CHAPTER 6
%----------------------------------------------------------------------------------------

\chapterimage{mangoes2.png} % Chapter heading image

\chapter{Architecture Compliance Analysis}

	\section{Specifications to which the System Adhered}
		\subsection{Web Interface}
			\begin{itemize}
				\item\textbf{HTML5} -
				HTML5 was used throughout the system as the markup for the Web interface. AngularJS was used in conjunction with HTML5 to get the dynamic functionality that we required.
				
				\item\textbf{CSS} -
				CSS was used for specific styling specifications that overwrote the default Bootstrap CSS styling. The colour scheme for the website was maintained throughout the system using external CSS files.
				
				\item\textbf{Google Maps JavaScript API Heatmap Layer} -
				We have started using the Google Maps API for handling the input of orchard block coordinates. However, it requires a public API key in order to have a certain amount of accesses a day. We haven't yet obtained this key as we are leaving the implementation for the map portion of the system for a later stage.
				
				\item\textbf{Bootstrap} -
				Bootstrap has been used for every page of our Web interface. We still need to iron the few issues related to the mobile-side of the application and we still need to fix our media queries. However, the majority of the Web interface is currently completely responsive.
				
				\item\textbf{AJAX} -
				We haven't come across a need to use AJAX so far but we are leaving it in our architecture requirements as we might still need to use it. If, at a later stage, we know for a fact that it is unnecessary, then we will remove it accordingly.
				
				\item\textbf{HTML5 Local Storage} -
				We haven't yet utilised HTML5 local storage as we are wanting to use it for storing user preferences, and at this stage, it is a nice-to-have functionality, which we will implement at a later stage.
				
				\item\textbf{Cookies} -
				We haven't yet utilised cookies as we are wanting to use it for storing user preferences, and at this stage, it is a nice-to-have functionality, which we will implement at a later stage.
				
				\item\textbf{Express.js} -
				Express.js has not been used in our system as of the time being. However, it will be required for server interaction for handling the sending of password recovery emails so it will be incorporated into the system at a later stage.
			\end{itemize}
		\subsection{Android Interface}
		The Android interface simply consists of six use cases, namely: login, logout, recover password, change password, view worker yield and update worker yield. Therefore, we have decided to implement the Android interface once the bulk of the Web interface's functionality is complete. This is due to the fact that the Android interface heavily relies on the Web interface. Thus, no implementation and testing has been done yet for the Android interface so an architecture compliance analysis can't be performed at this time. However, the technologies required for this interface have been listed below for convenience.
			\begin{itemize}
				\item{PhoneGap}				
				\item{Ionic}
			\end{itemize}
		\subsection{Both Systems}
			\begin{itemize}
				\item\textbf{JavaScript} -
				JavaScript has been used for all aspects of the system, as was expected. We used JavaScript within all the frameworks we have used in implementation thus far, and we have also used external JavaScript files to handle any processing that we couldn't achieve with the frameworks.
				
				\item\textbf{Sails.js} -
				We have used Sails.js extensively to implement the backend of the web interface. We have used most functions of the framework to link with the database, handle routing and handle the views of the web interface as well as implementing a RESTfull service.
				
				\item\textbf{AngluarJS} -
				We have incorporated AngularJS into our HTML code and used it in conjunction with the Sails.js framework to create a user-friendly and dynamic Web interface.
				
				\item\textbf{Node.js} -
				Sails.js handles all the Node.js functionality, such as the running of the server. So at this stage we have explictly incorporated Node.js, but it is implicitly being used within the Sails.js framework.				
				
				\item\textbf{MongoDB} -
				We are using MongoDB as our database system. We recently changed over from Neo4j to MongoDB so we are in the process of setting up MongoDB and changing our system over to this database change.
				
				\item\textbf{Unit.js} -
				We have yet to do unit tests on the functional components we have completed, but we are still considering this framework to assist us with them at a later stage.
				
				\item\textbf{Dagger} -
				We have yet to do unit tests on the functional components we have completed, but we are still considering this framework to assist us with them at a later stage.
				
				\item\textbf{Express.js} -
				As with Node.js, we have yet to explicitly incorporate this framework into our implementation. However, Sails.js does make use of it implicitly. It will also be required, in conjunction with Node.js, to handle the sending of password recovery emails to users.
				
				\item\textbf{Google App Engine Server} -
				We are still waiting for our client's financial manager to get back to us regarding whether they are willing to fund us to use this server. At this stage, it seems more than likely that we will be using this as the server as it allows for easy installation and integration of all the NPM packages used for implementation.			
			\end{itemize}	
	
%----------------------------------------------------------------------------------------
%	CHAPTER 7
%----------------------------------------------------------------------------------------

\chapterimage{litchiTree.png} % Chapter heading image

\chapter{Open Issues}

\section{Unit Testing}
	\begin{itemize}
		\item We are still uncertain about the way we are going to go about handling the unit testing for our system.
		\item We plan to set up a meeting with Fritz Solms to discuss our uncertainities and a way forward. Thus, at this stage we do not have any unit tests set up.
	\end{itemize}

\end{document}